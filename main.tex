% Created by: Hikari Hashida
% Date created: March 2020
% Date last modified: April 2020

% https://www.overleaf.com/learn/latex/How_to_Write_a_Thesis_in_LaTeX_(Part_1):_Basic_Structure
% https://www.sascha-frank.com/latex-font-size.html

\documentclass[12pt]{article}
\usepackage[utf8]{inputenc}
\usepackage{graphicx}
\usepackage{float}               % graphics float option
\usepackage{cancel}              % strike through
\usepackage[none]{hyphenat}      % prevent hyphenation
\usepackage{xcolor}
\usepackage{tcolorbox}			   % package for colours

\usepackage{amsmath}			      % package for mathematical equations
\usepackage{amssymb}			      % package for extra mathematical symbols

\usepackage{scalerel}            % for smaller subscripts

\usepackage{listings}             % for MATLAB scripts
\definecolor{codegreen}{rgb}{0,0.6,0}
\definecolor{codegray}{rgb}{0.5,0.5,0.5}
\definecolor{codepurple}{rgb}{0.58,0,0.82}
\definecolor{backcolour}{rgb}{0.95,0.95,0.92}

\lstdefinestyle{code_file}{
    backgroundcolor=\color{backcolour},   
    commentstyle=\color{codegreen},
    keywordstyle=\color{blue},
    numberstyle=\tiny\color{codegray},
    stringstyle=\color{codepurple},
    basicstyle=\ttfamily\footnotesize,
    breakatwhitespace=false,         
    breaklines=true,                 
    captionpos=b,                    
    keepspaces=true,                 
    numbers=left,                    
    numbersep=5pt,                  
    showspaces=false,                
    showstringspaces=false,
    showtabs=false,                  
    tabsize=2
}

\lstdefinestyle{output}{
    backgroundcolor=\color{backcolour},   
    commentstyle=\color{codegreen},
    keywordstyle=\color{magenta},
    stringstyle=\color{codepurple},
    basicstyle=\ttfamily\footnotesize,
    breakatwhitespace=false,         
    breaklines=true,                 
    captionpos=b,                    
    keepspaces=true,                              
    showspaces=false,                
    showstringspaces=false,
    showtabs=false,                  
    tabsize=2,
    numbers=none
}
\usepackage[margin=0.5in]{geometry}

\usepackage{hyperref}
\hypersetup{
    colorlinks=true,
    linkcolor=red,
    filecolor=magenta,      
    urlcolor=blue,
}

\graphicspath{ {images/} }

\renewcommand{\thefootnote}{\fnsymbol{footnote}}

% macro for creating boxed equation
\newcommand{\boxeq}[2]{
   \centering
   \begin{tcolorbox}[sharp corners=all, colback=white, colframe=red, text width = 10cm]
   \ifx&#2&
      \begin{align*} 
         #1
      \end{align*}
   \else
      \begin{align} 
         #1
      \end{align}
  \fi

   \end{tcolorbox}
   \raggedright\setlength{\parindent}{15pt}
}

% macro for creating new figure
\newcommand{\newfig}[4]{
   % \newdig{caption}{width-size}{figure-path}{labelling}
   \begin{figure} [H]
      \centering
      \includegraphics[width=#2\textwidth]{#3}  % size ratio and location
      \caption{#1}		                        % caption
      \label{fig:#4}                            % reference name
   \end{figure}
}

\begin{document}
   \begin{titlepage}
    \begin{center}
        \vspace*{1cm}
        \LARGE
        Report Title
 
        \vspace{0.5cm}
        \Large
        Course Code - Course Name
             
        \vspace{1.5cm}
      
        \includegraphics[width=0.5\textwidth]{UNSW-logo.png}

        \vspace{2cm}

        \normalsize
        I verify that the contents of this report are my own work
        \vspace{1cm}
        \begin{flushright}
        Firstname Lastname \\
        zID\\
        Date Month Year\\
        \end{flushright}
             
    \end{center}
\end{titlepage}

   \tableofcontents
   \input{sections/intro.tex}
   \input{sections/section1.tex}
   \input{sections/section2.tex}
   \input{sections/discuss.tex}
   \input{sections/conclude.tex}
   \begin{thebibliography}{9}
    \bibitem{my_label_name1} 
    J Doe.
    \textit{Laboratory Handout1}. 
    University of New South Wales, Lab Handout, Sydney, Year.

    \bibitem{my_label_name2} 
    J Doe.
    \textit{Laboratory Handout2}. 
    University of New South Wales, Lab Handout, Sydney, Year.
 \end{thebibliography}

 Refer to \href{https://www.google.com/url?sa=t&rct=j&q=&esrc=s&source=web&cd=&cad=rja&uact=8&ved=2ahUKEwjr1q_b0IzqAhVGbn0KHWkRDKAQFjABegQIDBAD&url=https%3A%2F%2Fieee-dataport.org%2Fsites%2Fdefault%2Ffiles%2Fanalysis%2F27%2FIEEE%2520Citation%2520Guidelines.pdf&usg=AOvVaw3WqzISVFjuCBM0Ti22r3oc}{this document} for IEEE citing guidelines.

   \appendix
   \section{Where do we begin?}

\subsection*{Bold}

\textbf{I am Bold}.

\subsection*{Italic}

\textit{I am Italic}.

\subsection*{Numbered bullet points}
\begin{enumerate}
    \item Use this to create your very own latex notes.
    \item Make sure you have installed necessary latex stuff.
    \item Have a fun journey!
\end{enumerate}

\subsection*{Bullet points}
\begin{itemize}
    \item To bullet,
    \item Or not to bullet,
    \item That is the question.
\end{itemize}


\section{Equations}

\subsection{Fraction}

\begin{equation}
    \frac{x}{2}+\frac{y}{2}=z
\end{equation}

\section{Figures and labelling}
\begin{verbatim}
    \newfig{caption}{width-size}{figure-path}{labelling}
\end{verbatim}

\newfig{Mars Rover}{0.5}{MarsRover.jpg}{fig:rover}

\boxeq{
    \frac{x}{2}+\frac{y}{2}=z
    \label{eqn:equation1}
}


I am referring to Eqn.\ref{eqn:equation1}.

\section{Coding Snippets}

\definecolor{mygreen}{rgb}{0,0.6,0}
\definecolor{mygray}{rgb}{0.5,0.5,0.5}
\definecolor{mymauve}{rgb}{0.58,0,0.82}

\lstset{showspaces=false,showstringspaces=false}

\begin{lstlisting}[language=C++,
    keywordstyle=\color{blue}\ttfamily,
    stringstyle=\color{red}\ttfamily,
    commentstyle=\color{mygreen}\ttfamily,
    breaklines=true]

    #include <iostream>

    using namespace;
    
    int main() {
        cout << "Hello world!" << endl;    
        return 0;
    }

\end{lstlisting}

\end{document}