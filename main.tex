% Created by: Hikari Hashida
% Date created: March 2020
% Date last modified: April 2020

% https://www.overleaf.com/learn/latex/How_to_Write_a_Thesis_in_LaTeX_(Part_1):_Basic_Structure
% https://www.sascha-frank.com/latex-font-size.html

\documentclass[12pt]{report}
\usepackage[utf8]{inputenc}
\usepackage{graphicx}
\usepackage{float}               % graphics float option
\usepackage{cancel}              % strike through
\usepackage[none]{hyphenat}      % prevent hyphenation
\usepackage{xcolor}
\usepackage{tcolorbox}			   % package for colours

\usepackage{amsmath}			      % package for mathematical equations
\usepackage{amssymb}			      % package for extra mathematical symbols

\usepackage{scalerel}            % for smaller subscripts

\usepackage{listings}             % for MATLAB scripts

\usepackage[margin=0.5in]{geometry}

\graphicspath{ {images/} }

\renewcommand{\thefootnote}{\fnsymbol{footnote}}

% macro for creating boxed equation
\newcommand{\boxeq}[2]{
   \centering
   \begin{tcolorbox}[sharp corners=all, colback=white, colframe=red, text width = 10cm]
   \ifx&#2&
      \begin{align*} 
         #1
      \end{align*}
   \else
      \begin{align} 
         #1
      \end{align}
  \fi

   \end{tcolorbox}
   \raggedright\setlength{\parindent}{15pt}
}

% macro for creating new figure
\newcommand{\newfig}[4]{
   % \newdig{caption}{width-size}{figure-path}{labelling}
   \begin{figure} [H]
      \centering
      \includegraphics[width=#2\textwidth]{#3}  % size ratio and location
      \caption{#1}		                        % caption
      \label{fig:#4}                            % reference name
   \end{figure}
}

\begin{document}
   \begin{titlepage}
    \begin{center}
        \vspace*{1cm}
        \LARGE
        Report Title
 
        \vspace{0.5cm}
        \Large
        Course Code - Course Name
             
        \vspace{1.5cm}
      
        \includegraphics[width=0.5\textwidth]{UNSW-logo.png}

        \vspace{2cm}

        \normalsize
        I verify that the contents of this report are my own work
        \vspace{1cm}
        \begin{flushright}
        Firstname Lastname \\
        zID\\
        Date Month Year\\
        \end{flushright}
             
    \end{center}
\end{titlepage}
   
   \chapter*{Preface}
   Some preface if you want.

   \tableofcontents

   \chapter{This is just the beginning}
   \input{chapters/chapter1.tex}

   \appendix
   \chapter{Some app-end-ix}
   \section{Appendix A}

Some appendix if you want.

\end{document}