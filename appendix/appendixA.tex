\section{Basics}

\subsection{Basic notations}
\lstnewenvironment{TeXlstlisting}{\lstset{language=[LaTeX]TeX}}{}

\lstset{style=output}

\begin{TeXlstlisting}
\textbf{I am Bold}.
\textit{I am Italic}.
\end{TeXlstlisting}
\noindent
\textbf{I am Bold}. \\
\textit{I am Italic}. \bigskip

\subsection{Bullet points}

\subsubsection{enumerate}
\begin{TeXlstlisting}
\begin{enumerate}
    \item Use this to create your very own latex notes.
    \item Make sure you have installed necessary latex stuff.
    \item Have a fun journey!
\end{enumerate}
\end{TeXlstlisting}

\begin{enumerate}
    \item Use this to create your very own latex notes.
    \item Make sure you have installed necessary latex stuff.
    \item Have a fun journey!
\end{enumerate}

\subsubsection{itemize}
\begin{TeXlstlisting}
\begin{itemize}
    \item To bullet,
    \item Or not to bullet.
\end{itemize}
\end{TeXlstlisting}

\begin{itemize}
    \item To bullet,
    \item Or not to bullet.
\end{itemize}


\section{Equations}

\subsection{Types}
\subsubsection{equation}

\begin{TeXlstlisting}
\begin{equation*}
    F = ma
\end{equation*}
\end{TeXlstlisting}

\begin{equation*}
    F = ma
\end{equation*}

\subsubsection{align}

\begin{TeXlstlisting}
\begin{align*}
    F   &= ma \\
        &= 3\times5 \\
        &= 15\ [N]
\end{align*}
\end{TeXlstlisting}

\begin{align*}
    F   &= ma \\
        &= 3\times5 \\
        &= 15\ [N]
\end{align*}

\subsubsection{gather}

\begin{TeXlstlisting}
\begin{gather*}
    F   = ma \\
       x = y+z \\
       z = x-y
\end{gather*}
\end{TeXlstlisting}

\begin{gather*}
    F   = ma \\
        = 3\times5 \\
        = 15\ [N]
\end{gather*}

\subsubsection{Numbering}
\begin{TeXlstlisting}
\begin{equation}
    F = ma 
\end{equation}

\begin{align}
    F   &= ma \nonumber \\
        &= 3\times5 \nonumber \\
        &= 15\ [N]
\end{align}

\begin{gather}
    F   = ma \\
        x = y+z \\
        z = x-y
\end{gather}
\end{TeXlstlisting}


\begin{equation}
    F = ma 
\end{equation}

\begin{align}
    F   &= ma \nonumber \\
        &= 3\times5 \nonumber \\
        &= 15\ [N]
\end{align}

\begin{gather}
    F   = ma \\
       x = y+z \\
       z = x-y
\end{gather}

\subsection{Fraction}

\begin{TeXlstlisting}
\begin{equation}
    \frac{x}{2}+\frac{y}{2}=z
\end{equation}
\end{TeXlstlisting}

\begin{equation}
    \frac{x}{2}+\frac{y}{2}=z
\end{equation}

\subsection{Matrices}

\begin{TeXlstlisting}
\begin{gather*}
    R = \begin{bmatrix}
        0& 0& 0 \\
        0& 0& 0 \\
        0& 0& 0
    \end{bmatrix}
\end{gather*}
\end{TeXlstlisting}

\begin{gather*}
    R = \begin{bmatrix}
        0& 0& 0 \\
        0& 0& 0 \\
        0& 0& 0
    \end{bmatrix}
\end{gather*}

\subsection{Notations}

\subsubsection{Cancel}

\begin{TeXlstlisting}
\begin{equation*}
    \bcancel{2}\frac{x}{\bcancel{2}}+\frac{\cancel{2y}}{\cancel{2y}}=x
\end{equation*}
\end{TeXlstlisting}

\begin{equation*}
    \bcancel{2}\frac{x}{\bcancel{2}}+\frac{\cancel{2y}}{\cancel{2y}}=x
\end{equation*}

\subsubsection{Colour}

\begin{TeXlstlisting}
\begin{equation*}
    2\textcolor{red}{\cancel{\textcolor{black}{x}}} (z+y) 
    = 5\cancel{\textcolor{red}{x}}
\end{equation*}
\end{TeXlstlisting}

\begin{equation*}
    2\textcolor{red}{\cancel{\textcolor{black}{x}}} (z+y) = 5\cancel{\textcolor{red}{x}}
\end{equation*}

\subsubsection{Braces}

\begin{TeXlstlisting}
\begin{equation*}
    d.A =  \underbrace{A+A+...+A}_\text{\textit{d}}
\end{equation*}
\end{TeXlstlisting}

\begin{equation*}
    d.A =  \underbrace{A+A+...+A}_\text{\textit{d}}
\end{equation*}


\section{Figures}

\subsection{Insert figures}

\begin{TeXlstlisting}
\newfig{Mars Rover}{0.5}{MarsRover.jpg}{fig:rover}    
\end{TeXlstlisting}
\newfig{Mars Rover}{0.5}{MarsRover.jpg}{fig:rover}

\subsection{Tables}

\begin{TeXlstlisting}
\begin{tabular} {    |>{\centering\arraybackslash}p{3cm}|
    |>{\centering\arraybackslash}p{3cm}|
    >{\centering\arraybackslash}p{3cm}|
    >{\centering\arraybackslash}p{3cm}|
    >{\centering\arraybackslash}p{3cm}|  
                }
\hline
\multicolumn{5}{|c|}{DH Parameter for SCARA manipulator} \\
\hline
Link \textit{i} & $a_i$ [m] & $\alpha_i$ [deg] & $d_i$ [m] & $\theta_i$ [deg] \\
\hline
1 & $a_1$   & 0     & 0     & *$\theta_1$\\
2 & $a_2$   & 180   & 0     & *$\theta_2$\\
3 & 0       & 0     & *$d_3$ & 0         \\
4 & 0       & 0     & $d_4$ & *$\theta_4$\\
\hline
\end{tabular}
\end{TeXlstlisting}

\begin{tabular} {    |>{\centering\arraybackslash}p{3cm}|
    |>{\centering\arraybackslash}p{3cm}|
    >{\centering\arraybackslash}p{3cm}|
    >{\centering\arraybackslash}p{3cm}|
    >{\centering\arraybackslash}p{3cm}|  
                }
\hline
\multicolumn{5}{|c|}{DH Parameter for SCARA manipulator} \\
\hline
Link \textit{i} & $a_i$ [m] & $\alpha_i$ [deg] & $d_i$ [m] & $\theta_i$ [deg] \\
\hline
1 & $a_1$   & 0     & 0     & *$\theta_1$\\
2 & $a_2$   & 180   & 0     & *$\theta_2$\\
3 & 0       & 0     & *$d_3$ & 0         \\
4 & 0       & 0     & $d_4$ & *$\theta_4$\\
\hline
\end{tabular} \bigskip


\section{Labelling and referencing}

\subsection{Referencing}
\begin{TeXlstlisting}
\boxeq{
    \frac{x}{2}+\frac{y}{2}=z
    \label{eqn:equation1}
}

I am referring to Eqn.\ref{eqn:equation1}.
\end{TeXlstlisting}

\boxeq{
    \frac{x}{2}+\frac{y}{2}=z
    \label{eqn:equation1}
}

I am referring to Eqn.\ref{eqn:equation1}.

\subsection{Footnote}

\begin{TeXlstlisting}
Time to peak is the time it takes to reach first peak of the output waveform\footnotemark. 
As we know from calculus, to find a stationary point (peak) of a curve, 
we need to find $t$ where $\dot{y}(t)=0$. \smallskip

\footnotetext{if its stable and oscillatory. Must be oscillatory as we're looking at a
 second order system. The amplitude of the output will increase exponentially if its 
 unstable, that means first peak won't be the maximum peak.}
\end{TeXlstlisting} \smallskip

Example: (see bottom of this page and notice the footnote). Time to peak is the time it takes to reach first peak of the output waveform\footnotemark. As we know from calculus, to find a stationary point (peak) of a curve, we need to find $t$ where $\dot{y}(t)=0$. \bigskip

\footnotetext{if its stable and oscillatory. Must be oscillatory as we're looking at a second order system. The amplitude of the output will increase exponentially if its unstable, that means first peak won't be the maximum peak.}

\section{Coding Snippets}

\subsection{Script file}

\begin{TeXlstlisting}
\lstset{style=code_file}

\begin{lstlisting}[language=C++]
    #include <iostream>

    using namespace std;
    
    int main() {
        cout << "Hello world!" << endl;    
        return 0;
    }

\end{lstlisting}
\end{TeXlstlisting}

\lstset{style=code_file}

\begin{lstlisting}[language=C++]
    #include <iostream>

    using namespace std;
    
    int main() {
        cout << "Hello world!" << endl;    
        return 0;
    }

\end{lstlisting}

\subsection{Output}
\lstset{style=output}

\begin{TeXlstlisting}
\lstset{style=output}

\begin{lstlisting}
Hello world!
\end{lstlisting}

\end{TeXlstlisting}

\lstset{style=output}

\begin{TeXlstlisting}
Hello world!
\end{TeXlstlisting}